% !TeX program = latexmk
\documentclass[11pt,a4paper]{ctexart}

\usepackage{homework}
\usepackage{biblatex}
\usepackage{hyperref}
\addbibresource{report.bib}

\title{PA1 Report}
\duedate{Mar 17, 2020}

% TODO your name and ID
\studentname{陈嘉杰}
\studentid{2017011484}

\usepackage{listings}
\lstset{basicstyle=\footnotesize\ttfamily}

\begin{document}

\maketitle


\section{第一部分}

第一部分的实验要求是实现朴素的 DPLL 算法。算法实现基本按照课件要求,只不过把递归形式改为了手动维护栈的形式,这样方便后续 backjump 的实现。代码中维护了如下的数据结构:

\begin{lstlisting}[language=c++]
struct LiteralInfo {
  // immutable
  std::vector<uint32_t> clauses;
  std::vector<uint32_t> clause_index;
  // mutable
  uint32_t cur_clauses;
  bool is_assigned;
#ifdef CDCL
  uint32_t unit_clause;
  uint32_t assign_depth;
#endif
};

struct ClauseInfo {
  // immutable
  std::vector<uint32_t> literals;
  // mutable
  uint32_t num_unassigned;
  bool is_satisfied;
};

enum ChangeType { TYPE_DECIDE, TYPE_IMPLIED };

struct Change {
  uint32_t assigned_literal;
  uint32_t removed_clauses_begin;
  ChangeType type;
};
\end{lstlisting}

LiteralInfo 记录了这个 literal 出现在的 clause 和对应的位置、当前出现在未满足 clause 的次数、是否已经赋值和用于 CDCL 的 implication graph 的边的记录。ClauseInfo 记录了这个 clause 中的各个 literal 、当前还未赋值的 literal 和是否已经满足。Change 记录了搜索树的一个结点,记录了此时是选择了一个 literal 还是因为 unit propagation 选择了一个 literal、目前赋值的 literal 并记录了因为赋值而被满足的 clause。

通过这些结构体,可以实现 DPLL 算法中需要用到的各个操作。考虑到这只是个小作业,并且公开的测试样例中数据量不是很大,没有做很深入的性能优化,比如通过位运算缩短在 clause 中寻找 unit clause 的时间(见 \cite{ahmed_implementation_nodate} )。

程序通过了公开的所有测例,并且我也额外从 \href{https://www.cs.ubc.ca/~hoos/SATLIB/benchm.html}{SATLIB - Benchmark Problems} 找到了一些 DIMACS 格式的测例,并加到了 dpll/tests 目录下,并额外手动够造了几个测例用于测试,一共 43 个测试样例,基于 Zhang Xinwei 和 Wang Yuanbiao 编写的脚本运行进行测试,在 Release 编译条件下都通过:

\begin{lstlisting}
0 unsat for 1 vars pass, time: 0.008457 ms
1 sat for 5 vars pass, time: 0.01464 ms
2 sat for 4 vars pass, time: 0.016514 ms
3 sat for 5 vars pass, time: 0.014561 ms
4 unsat for 20 vars pass, time: 0.083649 ms
5 sat for 12 vars pass, time: 0.04352 ms
6 sat for 20 vars pass, time: 0.207192 ms
7 unsat for 21 vars pass, time: 0.144755 ms
8 sat for 70 vars pass, time: 2.06928 ms
9 sat for 57 vars pass, time: 29.5635 ms
10 unsat for 90 vars pass, time: 1.63332 ms
11 sat for 42 vars pass, time: 0.447705 ms
12 unsat for 35 vars pass, time: 32.835 ms
13 sat for 45 vars pass, time: 2.92227 ms
14 unsat for 80 vars pass, time: 1.78398 ms
15 unsat for 50 vars pass, time: 0.315052 ms
16 unsat for 275 vars pass, time: 2.895 ms
17 sat for 163 vars pass, time: 1.63666 ms
18 unsat for 157 vars pass, time: 1.81069 ms
19 sat for 2 vars pass, time: 0.008153 ms
20 sat for 20 vars pass, time: 0.11078 ms
21 sat for 20 vars pass, time: 0.161564 ms
22 sat for 20 vars pass, time: 0.136904 ms
23 sat for 20 vars pass, time: 0.117698 ms
24 sat for 20 vars pass, time: 0.10649 ms
25 sat for 20 vars pass, time: 0.133034 ms
26 sat for 20 vars pass, time: 0.116897 ms
27 sat for 20 vars pass, time: 0.099551 ms
28 sat for 20 vars pass, time: 0.101954 ms
29 sat for 20 vars pass, time: 0.147096 ms
30 sat for 20 vars pass, time: 0.095605 ms
31 unsat for 50 vars pass, time: 1.46067 ms
32 unsat for 50 vars pass, time: 1.59957 ms
33 unsat for 50 vars pass, time: 1.35715 ms
34 unsat for 50 vars pass, time: 1.44918 ms
35 unsat for 50 vars pass, time: 1.92965 ms
36 sat for 100 vars pass, time: 216.186 ms
37 sat for 100 vars pass, time: 244.769 ms
38 sat for 100 vars pass, time: 13.4931 ms
39 sat for 100 vars pass, time: 1888.17 ms
40 sat for 100 vars pass, time: 143.096 ms
41 unsat for 9 vars pass, time: 0.082119 ms
42 sat for 7 vars pass, time: 0.033555 ms

total score: 43 / 43
\end{lstlisting}

\section{第二部分}

第二部分在第一部分的基础上实现了 CDCL,在代码中通过 CDCL 宏来进行控制,方便两个版本的对比。在遇到 冲突的时候,CDCL 算法会按照 implication graph 遍历寻找一个 cut,我实现的是找到所有入度为 0 的结点,也就是所有 Decide 而不是 Propagate 的结点。这样做的好处是可能能够 backjump 到较早的 decision level,坏处是这样的点可能很多,导致插入的 clause 不是很有效。在 \cite{kroening_decision_2008} 书中讲到一种根据图的最小割求出 conflict clause 的方法,由于生成的 literal 数量会比较少,所以应该会比我的算法得到更好的结果。

在找到这些结点后,把这些结点对应的 literal 的 negation 拼接起来成为一个新的 clause。由于此时这个 clause 所有 literal 都是 assigned 但没有一个是 true,所以它一定是 unsat 的,直接找到这些 literal 中 decision level 最大的一个,回退到 decide 它之前,此时就可以恢复正常的算法执行了,并且由于刚刚插入的这个 clause,会立即发生 unit propagate。

这里实际上也有两种实现的可能,一个可能是选择回到 decision level 最大的那一个的前一步,另一个可能是回到 decision level 最小的一个然后继续。第一种实现的优势是会减少走重复路径的可能,第二种实现的优势是 backjump 更远,并且可能会提前出现新的 unit propagate 路径,但在 conflict clause 很大的时候效果也不好。我没有做特别深入的比较研究,选择了前一种。

为了测试 CDCL ,我也构造了几个样例,然后比较了程序输出的 trace (通过 DEBUG 宏打开):

\begin{lstlisting}
\end{lstlisting}

\printbibliography

\end{document}
